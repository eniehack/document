\RequirePackage{ifluatex}
\ifluatex
	\documentclass[lualatex,ja=standard,12pt,a4j]{bxjsbook}
	\usepackage[ipa]{luatexja-preset}
\fi
\usepackage{here}
\usepackage{amsmath,amsfonts}
\usepackage{graphicx,xcolor}
\usepackage{longtable,multirow}
\usepackage{booktabs}
\usepackage[T1]{fontenc}
\usepackage[utf8]{inputenc}
\usepackage{textcomp}
\usepackage{url}

\title{入門 電算部}
\author{電算部員}

\begin{document}
	\tableofcontents
	\chapter{電算部入門}
		\section{この章の概要}
			この章では電算部とは何なのか、どう向き合っていくべきなのかについて説明する。
		\section{電算部は何であって何でないか}
			電算部をプログラミング教室と思っている人もいるかもしれないが、実際は全く違う。
			電算部は技術書の貸し出しや同じ志を持った人々との交流などを通して技術的な成長を支援する存在である。
			技術的な成長を支援する存在なだけであって、先輩がプログラミングを教えることはほとんどない。
			そのため、電算部を「部活」と捉えるのには語弊があり、「コミュニティ」と捉える方がより正確だ。
			したがって、部員ひとりひとりが能動的に活動することが求められる。
	\chapter{プログラミング入門}
		\section{この章の概要}
                本章はプログラミング初学者のための手引きである。
                これを読めば何をすればよいのかプログラミング言語をどのように勉強していくべきか、その後どうしていくべきかまで分かるように書いた(つもり)。
		\section{プログラミングを学ぶにあたって}
			\subsection{プログラミング学習の流れ}
				プログラミング言語を学習するのが始めてのときはもちろん、2回目以降も基本的には次の項目から順番に学んでいくことになる。
				\subsubsection{環境を整える}
					プログラムを書いたり実行したりするために、いくつかのソフトウェア\footnote{スマホでいうところのアプリ。}をPCにインストールする(=プログラミングのための環境を作る)。
	                
					なお、同じ環境を作る場合でも、PCの種類が違うと手順も変わる場合がある。
	                よって、インターネットの情報に頼って環境を作るときは、参考にする情報が自分の使っているパソコンに当てはめられるものかをよく確認しなければならない\footnote{分からないときは気軽にSlackできいてね。}。
					
				\subsubsection{プログラム言語の約束事(文法)を理解する}
	                それぞれの言語における、もっとも基本的な部分を理解する。
                  変数宣言や関数など、その言語を使っていくうえで欠かせない要素であるため、おろそかにしない。
				
				\subsubsection{基本的な制御構造を理解する}
					今言った「約束事」が人間の言葉でいう「文のつくり」ならば、「制御構造」とは「文章のつくり」である。
	                プログラムは条件文やループといった制御構造の組み合わせで成り立っているため、これもプログラムを書くうえでは欠かせない要素となる。
	                
				\subsubsection{データ(の集まり)の扱い方を理解する}
					プログラムは、与えられたデータから何かを生み出したり、データを変換したりするために書かれることが多い。
	                そして、データはその種類に応じて扱い方が異なる。したがって、それぞれについて扱い方を学ぶ必要がある。
					
				\subsubsection{プログラム言語によらないデータ処理の手法を理解する}
					なんのこっちゃ?\ と思うかもしれないが、要は「考え方を身に着けよう」という話である。
	                ここでは「アルゴリズム」と「データ構造」という2つのキーワードが登場する。
	                簡単に言うと、「アルゴリズム」とはデータを処理する手順で、「データ構造」とはデータを管理する仕組みだ。
	                
	                この項目は前段までの基本事項とは異なり、多くの言語に共通するものである\footonote{つまり「特定の」プログラム言語によらない、ということ。}。
	                その点ではこれが最も重要な学習事項といえるかもしれない。
					
				\subsubsection{問題の分野ごとに特有の手法を理解する}
	                ここでようやく、それぞれがプログラミングを通じて「やりたいこと」と関係する話になる。
	                
	                逆に言えば、ここまで来ると専門性が強まってくるので、先輩方でも知らないことが多くなってくる。
	                この段階まで来たなら、それぞれの分野のコミュニティーに参加するなどして情報をより広く共有するとよいだろう。
					
				\subsubsection{ライブラリの利用方法を理解する}
			        ライブラリとは、ある分野での汎用性の高い機能を色々な場所で使えるようにするためにひとつにまとめられたものだ。
			        ライブラリ単体では何もできず、ソフトウェアに取り込まれることによって初めて動く。
			        いわばドライバーやハンマーなどの工具のようなものだ。
			        
			        では、なぜライブラリを使うのだろうか。結論からいえば効率化である。
			        例えば、ゲームでは難しい物理演算が必要な場合がある。自分達で実装するのは面倒だが、これらをライブラリに任せることで、多少の不自由さと引き換えに物理演算のことをあまり考えることなく、
			        ゲームシステムをどうするかなど自分達のやりたいことに集中することができる。
			        
			        同じ言語で書かれ、機能の似たライブラリが複数ある場合もよくある。
			        ライブラリごとに対応する機能や、使いやすさ、ドキュメントのボリュームが違ってくるので、これらの観点で比べるといいだろう。 
			        
			        ライブラリを利用するにあたってはライセンスという利用するための条件が決まっていることが多い。
			        これを守らないと著作権などで訴えられる可能性があるので注意する。
			\subsection{分野別 おすすめプログラミング言語リスト}
				プログラミング言語には自分のやりたいことに向いているもの、向いていないものがある。
				
				表\ref{tbl:suggest_lang}に分野分野に向いている言語を挙げる。また、括弧内は有名なライブラリ名である。
				
				この表にやりたいことがない場合、Googleなどの検索エンジンで「【分野】 ライブラリ」などと検索してみるとよい。
				
				\begin{longtable}[H]{|c|c|}
					\caption{分野ごとのおすすめ言語}\label{tbl:suggest_lang}\\
					\toprule
					分野 & おすすめの言語 \tabularnewline
					\midrule
					\endhead
					\midrule
					\endfoot
					3D(\&2D)ゲーム開発 & C\#(Unity),\ C++(Unreal Engine) \tabularnewline
					2Dゲーム開発 & C/C++(SDL,cocos2d-x) \tabularnewline
					iPhoneアプリ開発 & Swift,\ Objective-C \tabularnewline
					Androidアプリ開発 & Java,\ Kotlin \tabularnewline
					Windows GUIアプリケーション開発 & \begin{tabular}{c}Visual Basic,\ C\#,\ C++(Qt),\\Python(Tkinter)\end{tabular} \tabularnewline
					Webサーバー & \begin{tabular}{c}Python,\ Ruby(Rails),\ PHP,\\Go,\ Java\end{tabular} \tabularnewline
					Webフロントエンド\footnote{ざっくり言えばWebページのこと。} & \begin{tabular}{c}HTML,\ CSS,\ JavaScript,\\TypeScript\end{tabular} \tabularnewline
					競技プログラミング & C++ \tabularnewline
					機械学習,ニューラルネットワーク & Python \tabularnewline
					データサイエンス & R,\ Python,\ Julia \tabularnewline
					Blenderプラグイン開発 & Python \\
					\bottomrule
				\end{longtable} 
		\section{イベント参加のススメ}
			\subsection{概要}
				群馬高専電算部では、部員がそれぞれ異なる分野を修めているため、部員に聞いてもわからないことが必ず出てくる。
                また、勉強しているうちに同じ志を持つ人との交流をしたいと思うときもあるだろう。
				
				そういうときは、イベントに参加することをおすすめする。
				特定の分野での知り合いができると質問ができるし、異なる分野の知り合いができても新しい発見ができて楽しい。
				しかも、イベント参加後は開発へのモチベーションが上がる。
				行って損はないのである。
				
				ここではイベントのへ参加方法、参加しやすいイベントの紹介をしていく。
			\subsection{イベントへの参加}
				\subsubsection{イベントを探す}
					まずはどんなイベントがあるかを調べる。イベントを探す専用のサイトが幾つかあるので紹介しよう。
					
					主なサイトとしてはconnpass\ <\url{https://connpass.com}>、ATND\ <\url{https://atnd.org}>、Doorkeeper\ <\url{https://doorkeeper.jp}>がある。
					
					どのサイトでもイベントを検索することができるので、自分の気になっているキーワード(例:Unity、Python、機械学習)を検索してみよう。
				\subsubsection{準備}
					イベントに参加するさいは名刺とPCを持っていくことをおすすめする。
					
					名刺を持っているとほかの人に名前を覚えてもらいやすくなるし、TwitterのIDを書いておくとtwitter上で交流が持てるかもしれない。
					
					また、登壇するさいには自前のPCが必須である。
                    また、参加者がPCで実況をする勉強会が(高専カンファレンスなど)一部存在する。
					
					また、登壇するさいには登壇時間に収まるスライドを制作し、練習しておく必要があるだろう。
					
					\subsubsection{その他}
						交通費や宿泊費などの、イベントへの参加費用は基本自己負担だが、出資してくれるイベントやサービス(サポーターズなど)もある。調べておくといいだろう。
			\subsection{おすすめのイベント}
				\subsubsection{高専カンファレンス}
					高専カンファレンスは主に高専生向けの勉強会である。
					2019年現在、全国各地で活発に開催されている。
					ちなみに、最近では勉強よりも参加者同士の交流に比重が置かれていることも多い。注意しておこう。
					
					高専生ひとりひとりの興味が違うため、いろいろな発表を聴くことができる。
					また、発表を聴くだけでなく、登壇することも可能である。
					高専生向けではあるが、高専OBやN高生、あるいは主催者の知り合いなど幅広い参加者がいることもある。
					
					最新情報は公式サイト<\url{http://kosenconf.jp}>から確認できる。
				\subsubsection{学生LT}
					中学生から大学生まで幅広い層が参加するLT(ライトニング・トーク)会である。
					主に東京、名古屋、大阪で開かれているが、最近では地方開催も増えてきている。
					
					こちらは高専カンファレンスと違い、あらゆる学生に向けて開かれたイベントであるため、普通高校の生徒や中学生、大学生、もちろん高専生とも、様々な人と交流することができる。
					一方で、高専カンファレンスと同じく登壇発表も可能である。
					
					最新情報は公式Twitter<\url{https://twitter.com/_student_lt}>で発信されている。
				\subsubsection{群馬高専IT勉強会}
					電算部の主催する学内向け勉強会。
					年4回開く予定である。
                    上2つのイベントに登壇する練習として使ってもいい。
					最新情報は電算部twitterと公式サイト<\url{https://nitgclt.connpass.com/}>を確認するといいだろう。
	\chapter{Slack入門}
		\section{この章の概要}
			電算部ではSlackを部内のコミュニケーションツールとして使用している。
            そこで、本章ではSlackの導入方法および簡単な使い方について説明する。
            
            SlackはPCとスマホの両方で利用できるが、OSによってそれぞれ導入手順などが異なるため、各項目ごとにそれぞれの場合における説明を行う。
            ここではWindows版、iOS版、Android版、ブラウザ版の4種類について説明し、MacOSおよびLinuxは割愛する。
            
        \section{導入手順}
        	Slackはアプリ版とブラウザ版に分かれる。
            アプリ版についてはインストール手順の説明を行い、ブラウザ版はインストール不要のため本項は省略する。
        	\begin{itemize} 
            	\item Windows\\
                	公式サイトのページ<\url{https://slack.com/downloads/}>からインストーラ(SlackSetup.exe)をダウンロードする。
                    このとき、PCのOSが32bitか64bitかを調べ、対応する方を選ぶ必要がある。
                    
                    SlackSetup.exeがダウンロードできたら、ダブルクリック、あるいは右クリックで「開く」または「管理者として実行」を選択して、インストールを開始する。
                	インストールが完了すればSlackが自動的に開くので、\ref{signin}節「サインイン」に移る。
                    
              	\item iOS・Android\\
                	iOSおよびAndroidのアプリ版は前述の<\url{https://slack.com/downloads/}>から各ストアへ飛んで、そのままインストールできる。
                    インストール完了後はアプリを開いて\ref{signin}節「サインイン」に移る。
                    
            \end{itemize}
		\section{サインイン\label{signin}}
        	インストールが完了したら、ワークスペースへのサインイン(=アカウント登録)を行う。
            なお、サインインの手順は全て共通。
            
            \begin{enumerate}
            	\item 招待を送ってもらう\\
                	電算部のワークスペースは完全招待制である。
                    そのため、既にメンバーとなっている部員(管理者でなくともよい)から招待コードを受け取る必要がある。
                    招待を送ってもらう側はメンバーの部員に自身のメールアドレスを渡す。
                    \footnote{アドレスはGmailを推奨する。普段使いするGoogleアカウントが無いなら、宗教上の理由が無い限りはまずアカウントを作って、それを使うようにする。}
                \item 招待を送る\\
                	アドレスを受け取った部員はSlackで「メンバーを招待する」からそのアドレスを入力し、招待コードを送る。
                    「メンバーを招待(する)」は端末によってアクセスの仕方が異なるため、次に示す。
                    \begin{itemize}
                    	\item Windowsアプリ/ブラウザ\\
                        	画面左上のワークスペース名をクリックして出てくるメニューの下方の「メンバーを招待」をクリック
                        \item iOS\\
                        	右端から左にスワイプして右サイドバーを開き、その中の「メンバーを招待」をタップ
                        \item Android\\
                        	右上の3つの点が並んだアイコンをタップし、下から2番目の「メンバーを招待する」をタップ
                    \end{itemize}
                \item 招待を承諾する
                	招待を送ってもらった側は、渡したアドレスにSlackからメールが来ていることを確認する。
                    メールが届いていたなら本文を開き、「今すぐ参加」をクリック/タップする。
                    届いていないときは迷惑メールの方に行っていないか、あるいは渡したアドレスに誤りがないか確認する。
            \end{enumerate}
            
   		\section{使い方}
        	\subsection{基本機能}
            	Slackは様々な機能を持つ反面、使いこなすには知らなければならないことがいくつかある。
                ここでは最も基本となるUIの説明を行う。
                \begin{itemize}
                	\item サイドバー(左)\\
                    	主にチャンネルを行き来するために使うメニューバー。
                        チャンネルの一覧はここに表示される。
                        
                        ブラウザ版やデスクトップアプリ版はデフォルトで左側に表示されているが、携帯端末では表示が隠れている。
                        そのため、スマホでは画面左端から右へスワイプすることでサイドバーを表示する。
                        なお、「画面左端」の判定がややシビアなので、スワイプの始点があまり端に寄っていないと開けないこともある。
                    \item 検索\\
                    	画面右上の検索ボックス、あるいは虫眼鏡のマークをクリック/タップすると、メッセージの検索ができる。
                        検索フィルターなどは開いてみれば書いてあるので省略。
                    \item 設定\\
                    	プロフィールやアプリの設定を行うことができる。
                        携帯端末では新規メンバーの招待もここから行う。
                        
                        設定を開くにはブラウザおよびデスクトップアプリでは左上のベルマークの左隣、携帯端末では検索ボックスの右の三点リーダめいたマーク\footnote{三点リーダ「・・・」を縦に並べたもの。}をそれぞれクリック/タップする。
                        iOSでは右端から左へスワイプすることで同じ画面を開くことが可能。
                \end{itemize}
            
            \subsection{チャンネル\label{channel}}
            	電算部のワークスペースには目的別にいくつかのチャンネルがある。
            	チャンネルの一覧は、左側のタブ(サイドバー)の「チャンネル」をクリック/タップして見ることができる。
            	一覧からチャンネルを選択するとそのチャンネルの内容や参加者が表示される。
            	参加したい場合は「チャンネルに参加する」をクリック/タップ。
            	チャンネルに参加すると、そのチャンネルで発言することができる。
            	また、既存のチャンネルとは別にトピックを立てたい場合は、チャンネルの一覧から「チャンネルを作成する」で気軽に立てられる。
            
            	次に主なチャンネルの一覧を示す。
            
            	\begin{itemize}
            		\item 部内連絡\\
                        ワークスペースに参加すると自動的に追加されるチャンネルになっている。
                        ここでは主要な連絡事項を伝える。
                        また、特に周知すべきことは@everyoneでデスクトップあるいはポップアップ通知が行く。
                	\item 雑談\\
                    	雑談をする。
                        ただしチャットルームではなく、むしろ部内連絡で言うほどではないまでも、特別に話したいことを呟くチャンネルとして機能している(と思われる)ので注意。
                        電算部が利用しているフリープランは10000件しかメッセージを保持できないため、あまりにも電算部や情報系に関係ない話題はTwitter等へ。
                    \item help-c\\
                    	C言語の質問チャンネル。
                        とくにJ科は1年の後期から3年までC言語を用いてプログラミングの基礎を学習するため、課題などで分からないときはここで先輩たちに訊くことができる。
                        C言語を自分の力だけで学ぶのは大変なので積極的に利用してほしい。
                        
                        また、他の言語についても同じようなチャンネルが開設されている。
                    \item 欲しい\\
                    	部費で欲しいものを呟くチャンネル。
                        しかし、年度始めに予算の配分が終わると追加購入は基本的に不可能なため、一年に一度しか機能しない。
                        先にも述べたが10000件しかメッセージを保持できないため、欲しいものを投げてもタイミングによっては流れてしまう可能性がある。
                    \item profile\\
                    	自己紹介を投げるチャンネル。顔と名前が一致しなければあまり意味がないので、積極的に参加しなくともよいと思われる。
            	\end{itemize}
            
            	以上に示した他にも、ジャンル別に様々なチャンネルが存在するので、一度はチャンネルの一覧を見て確認することを推奨する。
                
   			\subsection{ダイレクトメッセージ\label{directmessage}}
            	\ref{channel}のチャンネルは、ワークスペース内の誰でも参加できる、オープンな情報共有の場である。
                それに対して、ダイレクトメッセージは相手を絞って対話できる、クローズドな情報共有の場といえる。
                ダイレクトメッセージを開始するにはサイドバーから「ダイレクトメッセージ」の横の+アイコンをクリック/タップ。
                メッセージには自分の他に最大8人まで参加可能である。
            
            \subsection{メッセージの主要機能}
            	メッセージを送る際に使えるいくつかの機能を列挙する。
                
                ざっくりと紹介するに留めるため、細かい機能については各自調べてほしい。
                
                \begin{itemize}
                	\item メンション\\
                    	メンションは特定の相手、あるいは全員に対して自分のメッセージを通知するための機能である。
                        つまりはTwitterの「@ツイート」で、使い方も一緒。
                        @に続けて対象の表示名を入力、あるいは@を打って上に表示される選択肢から選ぶことで、その対象に通知が行く。
                        複数選択も可能な他、「@everyone」とすることで全員に通知できる。
                        ただし、メンバー全員が参加していないチャンネルでは使えないため、代わりに「@channel」を用いる。
                        しかし使いすぎると(主に音ゲーマーから)ひんしゅくを買うのでほどほどに。
                    \item スレッド\\
                    	自分や他のメンバーが投稿したメッセージからスレッドを開始することができる。
                        通常のメッセージとは異なり、タイムライン上には「n件の返信」とだけ表示される。
                        この機能はメッセージに対するコメントなど、タイムラインで全員に見せるほどでもない事柄を相手に伝えるのに使われる。
                        ただしこれはTwitterにおけるリプライのようなもので、スレッドに加わっていないメンバーでも閲覧できるため、内緒話には向いていない。
                        プライベートな話には前項\ref{directmessage}のダイレクトメッセージを使うとよい。
                    \item 絵文字・リアクション\\
                    	絵文字が利用可能。
                        メッセージの中で使えるほか、メッセージに対する「リアクション」に用いられる。
                        リアクションは、反応を残したいがわざわざスレッドを残すほどのことでもないときに使う。
                    \item メッセージの共有\\
                    	自分や他のメンバーが投稿したメッセージを、他のチャンネルや他のメンバーに共有できる。
                    \item ファイルのアップロード\\
                    	ドラッグ&ドロップやメッセージ欄左側の+ボタンから画像などのファイルをアップロードできる。
                        ワークスペース全体を通したファイルの容量には限界があるので、これもやりすぎないように。
                    \item コードのスニペット\\
                    	プログラムのソースコードをスニペットという形で共有できる。
                        スニペットとして共有すると色々便利なので、スクショやコードのファイルを上げるよりもこちらを使うとよい。
                \end{itemize}
                
            	ここに挙げた以外にも様々な機能が存在するため、調べて使ってみるとよい。

	\chapter{電算部wiki入門}
    	\section{この章の概要}
        	マニュアルの最後として、今年度より立ち上げた電算部wikiを紹介する。
            2019年4月の段階では未だローカルサーバに留めているが、いずれは部内Wi-Fiに接続した全ての端末で閲覧可能にする予定である。
            ただし、一般的なwikiと異なり、編集だけでなく閲覧そのものにアカウントが必要となる。
            そのアカウントについてもここで説明する。
            
        \section{電算部wikiとは何か}
        	電算部wikiとはすなわち、電算部員による電算部員のための電算部についてまとめたwikiである。
            
            そもそも部活動は、一般に存在するクラブ活動と異なり、1年ごとにその体制が変わっていくものだ。
            特に電算部は、各々の部員がばらばらに活動しているために、他の部に比べてノウハウが伝わりにくくなっている。
            既に導入されているSlackはあくまで情報の伝達をスムーズに行うためのツールであり、情報の保存には向いていない。
            
            そこで、部員が情報を次の世代へ伝えるためのシステムとして、部内wikiが作成されたのである。
            
        \section{アカウントについて}
        	当wikiを利用するためには、個人のアカウントが必要となる。
            これは個人の特定につながる情報や、部室および学校のセキュリティに関する情報の漏洩を避けるための措置である。
            
            \subsection{アカウントの作成}
            	作成と書いたが、アカウント自体はこちらが予め作成したものを配布する予定である。
                ただし、配布するIDとパスワードは機械的に決定されたものであり、さらにそのままではアカウントの作成者がパスワードを把握している状態になってしまう。
                
                そのため、配布されたIDとパスワードは各自で設定し直さなければならない。
                設定方法は次の通りである。
                
                \begin{enumerate}
                	\item 配布されたIDとパスワードでログインする
                    \item 右上の「ユーザー情報の更新」をクリック
                    \item 開いたページでID\footnote{半角英数だけでなく日本語も使える。}、パスワードが変更できる
                \end{enumerate}
                
		\section{利用のし方}
        	アカウントを持つユーザーは、wiki内のあらゆる記事の閲覧と編集が可能である。
            先に述べたように、現時点ではサーバを置いているPCでのアクセスに限定されているが、今後は部内Wi-Fiに接続していればどの端末でもアクセスできる予定となっている。
            
            ここでは、閲覧と編集それぞれについての手引きをここに示す。
            
            \subsection{閲覧する}
            	閲覧、つまり記事を読むことである。
                簡単だが、とても重要なことだ。
                wikiは編集する人だけいても、誰も見なければ意味が無いのだ。
                だから一度は目を通してもらいたい。
                
                記事の内容にも触れておくと、メインは年中行事やイベントに関するものと、部の運営に関するものになる。
                
                イベントの記事にはそのイベントの概要と、参加・運営のハウツーが記されている。
                工華祭
                \footnote{こうかさい。群馬高専の文化祭。隔年開催で、本科の全学級と部活動・愛好会がそれぞれ催し物や展示を行う。}
                や文発
                \footnote{ぶんぱつ。関東の高専から文化部が集まり、発表や展示を行う。毎年開催され、会場は運営する高専かその近傍になる。}
                といったイベントのときに、例年どのように進められたかを確認するのに使われることを想定している。
                
                部の運営はそのまま、各年度の予算および部員から集めた部費の使途などの情報を掲載し、運営に役立てることが目的の記事になる。
                
                また、現在は存在しないが、部員それぞれの取り組んでいる分野に関する記事も作っていく(作ってもらう)予定だ。
                
            \subsection{編集する}
            	編集、つまり記事を書くことである。
                閲覧よりも難しいが、大したことはない。
                
                理由はいくつかある。
                まず1つに、使っている言語が非常に簡単なことが挙げられる。
                
                このwikiで使っているのはmarkdownと言って、HTML
                \footnote{ウェブページを作るためのマークアップ言語の1つ。
                HTMLを使って書くと、文字列と記号の組み合わせでウェブページに様々な機能を追加できる。}
                を簡単に書くための言語だ。
                簡単というのはつまり、簡単な記号で文章の表示や構造を変えられる、ということである。
                簡単なぶん機能も限定されているが、wikiのようなほぼプレーンテキストのメディアには好都合といえる。
                
                次に、部内向けであること。
                
                一般的なwikiはあらゆる閲覧者を想定して書く必要があるため、一つ一つの表現に気を配る必要がある。
                一方で部内wikiは、閲覧者が電算部員に限定されることから、内輪でしか通じないような表現も使える(あまり推奨はしないが)。
                電算部wikiは気軽に編集できるので、何か書き残したいことがあったらどしどし追記していってもらいたい。
                
                特に、それぞれの取り組んでいる分野について、あるいは参加したイベントなど、気付きや体験談を残してもらえると非常に助かる。
                
                編集方針やmarkdown記法の使い方については、部内オンライン化までに整備する予定だ。
                

\end{document}
