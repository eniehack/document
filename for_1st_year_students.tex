\RequirePackage{ifluatex}
\ifluatex
	\documentclass[lualatex,ja=standard,12pt,a4j]{bxjsbook}
	\usepackage[ipa]{luatexja-preset}
\fi
\usepackage{here}
\usepackage{amsmath,amsfonts}
\usepackage{graphicx,xcolor}
\usepackage{longtable,multirow}
\usepackage{booktabs}
\usepackage[T1]{fontenc}
\usepackage[utf8]{inputenc}
\usepackage{textcomp}

\title{入門 電算部}
\author{電算部員}

\begin{document}
	\tableofcontents
	\chapter{電算部入門}
		\section{この章の概要}
			この章では電算部とな何なのか、どう向き合っていくべきなのかについて説明する。
		\section{電算部は何であって何でないか}
			電算部をプログラミング教室と思っている人もいるかもしれないが、実際は全く違う。
			電算部は技術書の貸し出しや同じ志を持った人々との交流などを通して技術的な成長を支援する存在である。
			技術的な成長を支援する存在なだけであって、先輩がプログラミングを教えることはほとんどない。
			そのため、電算部を「部活」と捉えるのには語弊があり、「コミュニティ」と捉えるのが正しく、
			また、部員ひとりひとりが能動的に活動することが必要である。
	\chapter{プログラミング入門}
		\section{この章の概要}
                本章はプログラミング初学者のための手引きである。
                これを読めば何をすればよいのか分かるように書いた(つもり)。
                
		\section{プログラミングを学ぶにあたって}
			\subsection{プログラミング学習の流れ}
			\begin{enumerate}
				\item 環境を整える\\
				プログラムを書いたり実行したりするために、いくつかのソフトウェア(スマホでいうアプリケーション)をPCにインストールする(=プログラミングのための環境を作る)。
                
				なお、同様の環境を構築する場合でも、その手順は計算機の機種ごと、OSごとに異なる。
                インターネットの情報に頼って環境構築をする際は、その説明が自分の場合に当てはめられるかをよく確認しなければならない。
				
				\item プログラム言語の約束事(文法)を理解する\\
                それぞれの言語における、もっとも基本的な部分を理解する。
                変数宣言や関数など、その言語を使用するうえで欠かせない要素であるため、疎かにしない。
			
				\item 基本的な制御構造を理解する\\
				前段における「約束事」が人間の言語でいう「文」の構造ならば、「制御構造”とは「文章」の構造のことである。
                プログラムは、条件文やループといった制御構造の組み合わせで成り立っているため、これもプログラムを書くうえでは不可欠な要素となる。
                
				\item データ(の集まり)の扱い方を理解する\\
				プログラムは、与えられたデータから何かを生み出したり、データを変換したりするために書かれることが多い。
                そして、データはその種類に応じて扱い方が異なる。したがって、それぞれについて扱い方を学ぶ必要がある。
				
				\item プログラム言語によらないデータ処理の手法を理解する\\
				なんのこっちゃ? と思うかもしれないが、要は「考え方を身に着けよう」という話である。
                ここでは「アルゴリズム」と「データ構造」という2つのキーワードが登場する。
                簡単に言うと、「アルゴリズム」とはデータを処理する手順で、「データ構造」とはデータを管理する仕組みだ。
                
                この項目は前段までの基本事項とは異なり、多くの言語に共通するものである。
                その点ではこれが最も重要な学習事項といえるかもしれない。
				
				\item 問題の分野ごとに特有の手法を理解する\\
                ここでようやく、それぞれがプログラミングを通じて「やりたいこと」と関係する話になる。
                
                逆に言えば、ここまで来ると専門性が強まってくるので、先輩方でも知らないことが多くなってくる。
                この段階まで来たなら、それぞれの分野のコミュニティーに参加するなどして情報をより広く共有するとよいだろう。
				
				\item ライブラリの利用方法を理解する\\
				%プログラム言語や開発環境ごとに用意されている
                %わからないので誰か書いて
                
			\end{enumerate}
	\chapter{Slack入門}
		\section{この章の概要}
			電算部ではSlackを部内のコミュニケーションツールとして使用している。
            そこで、本章ではSlackの導入方法および簡単な使い方について説明する。
            
            SlackはPCとスマホの両方で利用できるが、OSによってそれぞれ導入手順などが異なるため、各項目ごとにそれぞれの場合における説明を行う。
	\chapter{Wiki管理入門}
		\section{この章の概要}
		\section{アカウントの作成}
	\chapter{部室管理について}
		\section{この章の概要}
		\section{鍵の管理}
		\subsection{鍵の借り方}
		\subsection{鍵の返し方}
\end{document}