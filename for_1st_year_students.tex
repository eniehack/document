\RequirePackage{ifluatex}
\ifluatex
	\documentclass[lualatex,ja=standard,12pt,a4j]{bxjsbook}
	\usepackage[ipa]{luatexja-preset}
\fi
\usepackage{here}
\usepackage{amsmath,amsfonts}
\usepackage{graphicx,xcolor}
\usepackage{longtable,multirow}
\usepackage{booktabs}
\usepackage[T1]{fontenc}
\usepackage[utf8]{inputenc}
\usepackage{textcomp}

\title{入門 電算部}
\author{電算部員}

\begin{document}
	\tableofcontents
	\chapter{プログラミング入門}
		\section{この章の概要}
		\section{プログラミングを学ぶにあたって}
			\subsection{プログラミング学習の流れ}
			\begin{enumerate}
				\item 環境を整える\\
				プログラムを書いたり、実行したりするためには、いくつかのソフトウェア(スマホでいうアプリケーション)をPCにインストールする(=プログラミングのための環境を作る)必要があります。
                
				なお、同様の環境を構築する場合でも、その手順は計算機の機種ごと、OSごとに異なります。
				
				\item プログラム言語の約束事(文法)を理解する\\
				変数宣言
			
				\item 基本的な制御構造を理解する\\
				条件分岐、繰返し
				
				\item データ(の集まり)の扱い方を理解する\\
				配列など
				
				\item プログラム言語によらないデータ処理の手法を理解する\\
				データ構造とアルゴリズム
				
				\item 問題の分野ごとに特有の手法を理解する\\
				例:画像処理、人工知能
				
				\item ライブラリの利用方法を理解する\\
				プログラム言語や開発環境ごとに用意されている
			\end{enumerate}
	\chapter{Slack入門}
		\section{この章の概要}
			電算部ではSlackを部内のコミュニケーションツールとして使っています。
            そこで、本章においてSlackの導入方法および簡単な使い方について説明します。
            SlackはPCとスマホの両方で利用でき、それぞれ導入手順などが異なるため別々に説明します。
		\section{Windows PC}
		\section{Android スマホ/タブレット}
		\section{iOS スマホ/タブレット}
        %上記以外で
	\chapter{Wiki管理入門}
		\section{この章の概要}
		\section{アカウントの作成}
	\chapter{部室管理について}
		\section{この章の概要}
		\section{鍵の管理}
		\subsection{鍵の借り方}
		\subsection{鍵の返し方}
\end{document}