\RequirePackage{ifluatex}
\ifluatex
	\documentclass[lualatex,ja=standard,12pt,a4j]{bxjsbook}
	\usepackage[ipa]{luatexja-preset}
\fi
\usepackage{here}
\usepackage{amsmath,amsfonts}
\usepackage{graphicx,xcolor}
\usepackage{longtable,multirow}
\usepackage{booktabs}
\usepackage[T1]{fontenc}
\usepackage[utf8]{inputenc}
\usepackage{textcomp}
\usepackage{url}

\title{入門 電算部}
\author{電算部員}

\begin{document}
	\tableofcontents
	\chapter{電算部入門}
		\section{この章の概要}
			この章では電算部とな何なのか、どう向き合っていくべきなのかについて説明する。
		\section{電算部は何であって何でないか}
			電算部をプログラミング教室と思っている人もいるかもしれないが、実際は全く違う。
			電算部は技術書の貸し出しや同じ志を持った人々との交流などを通して技術的な成長を支援する存在である。
			技術的な成長を支援する存在なだけであって、先輩がプログラミングを教えることはほとんどない。
			そのため、電算部を「部活」と捉えるのには語弊があり、「コミュニティ」と捉えるのが正しく、
			また、部員ひとりひとりが能動的に活動することが必要である。
	\chapter{プログラミング入門}
		\section{この章の概要}
                本章はプログラミング初学者のための手引きである。
                これを読めば何をすればよいのかプログラミング言語をどのように勉強していくべきか、その後どうしていくべきかまで分かるように書いた(つもり)。
		\section{プログラミングを学ぶにあたって}
			\subsection{プログラミング学習の流れ}
			\subsubsection{環境を整える}
				プログラムを書いたり実行したりするために、いくつかのソフトウェア(スマホでいうアプリケーション)をPCにインストールする(=プログラミングのための環境を作る)。
                
				なお、同様の環境を構築する場合でも、その手順は計算機の機種ごと、OSごとに異なる。
                インターネットの情報に頼って環境構築をする際は、その説明が自分の場合に当てはめられるかをよく確認しなければならない。
				
			\subsubsection{プログラム言語の約束事(文法)を理解する}
                それぞれの言語における、もっとも基本的な部分を理解する。
                変数宣言や関数など、その言語を使用するうえで欠かせない要素であるため、疎かにしない。
			
			\subsubsection{基本的な制御構造を理解する}
				前段における「約束事」が人間の言語でいう「文」の構造ならば、「制御構造”とは「文章」の構造のことである。
                プログラムは、条件文やループといった制御構造の組み合わせで成り立っているため、これもプログラムを書くうえでは不可欠な要素となる。
                
			\subsubsection{データ(の集まり)の扱い方を理解する}
				プログラムは、与えられたデータから何かを生み出したり、データを変換したりするために書かれることが多い。
                そして、データはその種類に応じて扱い方が異なる。したがって、それぞれについて扱い方を学ぶ必要がある。
				
			\subsubsection{プログラム言語によらないデータ処理の手法を理解する}
				なんのこっちゃ? と思うかもしれないが、要は「考え方を身に着けよう」という話である。
                ここでは「アルゴリズム」と「データ構造」という2つのキーワードが登場する。
                簡単に言うと、「アルゴリズム」とはデータを処理する手順で、「データ構造」とはデータを管理する仕組みだ。
                
                この項目は前段までの基本事項とは異なり、多くの言語に共通するものである。
                その点ではこれが最も重要な学習事項といえるかもしれない。
				
			\subsubsection{問題の分野ごとに特有の手法を理解する}
                ここでようやく、それぞれがプログラミングを通じて「やりたいこと」と関係する話になる。
                
                逆に言えば、ここまで来ると専門性が強まってくるので、先輩方でも知らないことが多くなってくる。
                この段階まで来たなら、それぞれの分野のコミュニティーに参加するなどして情報をより広く共有するとよいだろう。
				
			\subsubsection{ライブラリの利用方法を理解する}
		        ライブラリとは、ある分野での汎用性の高い機能を色々な場所で使えるようにするためにひとつにまとめられたものだ。
		        ライブラリ単体では何もできず、ソフトウェアに取り込まれることによって初めて動く。
		        いわばドライバーやハンマーなどの工具のようなものだ。
		        
		        では、なぜライブラリを使うのだろうか。結論からいえば効率化である。
		        例えば、ゲームでは難しい物理演算が必要な場合がある。自分達で実装するのは面倒だが、これらをライブラリが任せることで多少の不自由さと引き換えに物理演算のことあまりを考えることなく、
		        ゲームシステムをどうするかなど自分達のやりたいことに集中することができる。
		        
		        同じ言語で書かれ、機能の似たライブラリが複数ある場合もよくある。
		        ライブラリごとに対応する機能や、使いやすさ、ドキュメントのボリュームが違ってくるので、これらの観点で比べるといいだろう。 
		        
		        ライブラリを利用するにあたってはライセンスという利用するための条件が決まっていることが多い。
		        これを守らないと著作権などで訴えられる可能性があるので気をつけよう。           
		\section{イベント参加のススメ}
			\subsection{概要}
				部員ひとりひとりが別々の分野を極めているのが当たり前である電算部では、部員に聞いてもわからないことが必ず出てくる。また、同じ志を持つ人との交流をしたいと思うときもある。
				
				そういうときはイベントに参加することをおすすめする。
				特定の分野での知り合いができると質問ができるし、異なる分野の知り合いができても新しい発見ができて楽しい。
				しかも、イベント参加後は開発へのモチベーションが上がる。
				行って損はないのである。
				
				ここではイベントのへ参加方法、参加しやすいイベントの紹介をしていく。
			\subsection{イベントへの参加}
				\subsubsection{イベントを探す}
					まずはイベントを探そう。イベントを探す専用のサイトが幾つかあるので紹介しよう。
					
					主なサイトとしてはconnpass<\url{https://connpass.com}>、ATND<\url{https://atnd.org}>、Doorkeeper<\url{https://doorkeeper.jp}>がある。
					
					どのサイトでもイベントを検索することができるので自分の気になっているキーワード(例:Unity、Python、 機械学習など)を検索してみよう。
				\subsubsection{準備}
					イベントに参加するさいは名刺とPCを持っていくことをおすすめする。
					
					名刺を持っているとほかの人に名前を覚えてもらいやすくなるし、TwitterのIDを書いておくとtwitter上で交流が持てるかもしれない。
					
					登壇するさいには自前のPCが必須である。また、参加者がPCで実況をする勉強会が(高専カンファレンスなど)一部存在する。
					
					また、登壇するさいには登壇時間に収まるスライドを制作し、練習しておく必要があるだろう。
					
					\subsubsection{その他}
						交通費や宿泊費などの、イベントへの参加費用は基本自己負担がだが、出資してくれるイベントやサービス(サポーターズなど)もある。調べておくといいだろう。
			\subsection{おすすめのイベント}
				\subsubsection{高専カンファレンス}
					高専カンファレンスは主に高専生向けの勉強会である。
					2019年現在、全国各地で活発に開催されている。
					ちなみに、最近では勉強よりかは参加者同士の交流に比重が置かれていることも多い。注意しておこう。
					
					高専生ひとりひとりの興味が違うため、いろいろな発表を聴くことができる。
					また、発表を聴くだけでなく、登壇することも可能である。
					高専生向けとはいうが高専OBやN高生、主催者の知り合いなど幅広い参加者がいることもある。
					
					最新情報は公式サイト<\url{http://kosenconf.jp}>を確認するといいだろう。
				\subsubsection{学生LT}
					中学生から大学生まで幅広い層が参加するLT(ライトニング・トーク)会である。
					主に東京、名古屋、大阪で開かれているが、最近では地方開催も増えてきている。
					
					高専生向けでなく、学生に向かって開かれたイベントであるため、普通高校の生徒や中学生、大学生、もちろん高専生とも、いろいろん人と交流することができる。
					また、高専カンファレンスと同じく、登壇することも可能である。
					
					最新情報は公式Twitter<\url{https://twitter.com/_student_lt}>を確認するといいだろう。
				\subsubsection{群馬高専IT勉強会}
					電算部の主催する学内向け勉強会。
					年4回開く予定である。上2つのイベントに登壇する練習として使ってもいい。
					最新情報は電算部twitterと公式サイト<\url{https://nitgclt.connpass.com/}>を確認するといいだろう。
	\chapter{Slack入門}
		\section{この章の概要}
			電算部ではSlackを部内のコミュニケーションツールとして使用している。
            そこで、本章ではSlackの導入方法および簡単な使い方について説明する。
            
            SlackはPCとスマホの両方で利用できるが、OSによってそれぞれ導入手順などが異なるため、各項目ごとにそれぞれの場合における説明を行う。
\end{document}